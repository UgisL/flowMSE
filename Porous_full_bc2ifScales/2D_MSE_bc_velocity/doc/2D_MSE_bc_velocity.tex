\NeedsTeXFormat{LaTeX2e}
\documentclass[12pt,a4paper]{article}

\usepackage[english]{babel}
\usepackage[latin1]{inputenc}
\usepackage[T1]{fontenc}

\usepackage{hyperref}
\hypersetup{
    colorlinks=false,
    pdfborder={0 0 0},
}

%\usepackage[lite,subscriptcorrection,slantedGreek,nofontinfo]{mtpro2}
%\usepackage{microtype}

\usepackage{parskip}
%\usepackage{units}
\usepackage{xspace}
\usepackage{fancyhdr}
\usepackage{amsmath}
%\usepackage{amssymb}
\usepackage{graphicx}
%\usepackage{units}
%\usepackage[section]{placeins}
\usepackage{cite}
\usepackage{subfigure}

% Reduce margins
\usepackage{a4wide}
% Header style
\pagestyle{fancy}
\fancyfoot[L]{\today}
\fancyfoot[C]{}
\fancyfoot[R]{Ugis Lacis}
\fancyhead[L]{Documentation}
\fancyhead[C]{2D\_MSE\_bc\_velocity}
\fancyhead[R]{Page \thepage}

\usepackage{times}
%\usepackage{mathptm}
\usepackage{longtable}
\usepackage{multirow}

\usepackage{float}  %My added packages [UgisL]
\usepackage{verbatim}
\usepackage{url}

% Code inclusion
\usepackage{listings}
\usepackage{color}

\definecolor{dkgreen}{rgb}{0,0.6,0}
\definecolor{gray}{rgb}{0.5,0.5,0.5}
\definecolor{mauve}{rgb}{0.58,0,0.82}

\lstset{language=C++,
  aboveskip=3mm,
  belowskip=3mm,
  xleftmargin=\parindent,
  showstringspaces=false,
  columns=flexible,
  basicstyle={\scriptsize\ttfamily},
  numbers=none,
  numberstyle=\tiny\color{gray},
  keywordstyle=\color{blue},
  commentstyle=\color{dkgreen},
  stringstyle=\color{mauve},
  breaklines=true,
  breakatwhitespace=true,
  tabsize=3
}


\newcommand{\pd}{\partial}
\newcommand{\pdt}{\partial_t}

% Package for doing IF clauses withing newcommand
\usepackage{xstring}
% Commands for strain and ordering parameters
\newcommand{\str}[3]{\varepsilon^{#1}_{#2} \left( #3 \right)}
\newcommand{\ord}{\epsilon}
\newcommand{\vare}[2]{ #2^{(#1)} }
%\newcommand{\orde}[2]{\epsilon^{#1} #2^{(#1)} }
\newcommand{\orde}[2]{%
    \IfEqCase{#1}{%
        {0}{  #2^{(#1)} }%
        {1}{ \epsilon #2^{(#1)} }%
    }[ \epsilon^{#1} #2^{(#1)} ]%
}
\newcommand{\ordep}[2]{%
    \IfEqCase{#1}{%
        {0}{  #2^{(#1)} }%
        {1}{ + \epsilon #2^{(#1)} }%
    }[ + \epsilon^{#1} #2^{(#1)} ]%
}
\newcommand{\ordest}[1]{\mathcal{O} \left( #1\right)}
% Package for for loops
\usepackage{pgffor}
% Command to print the complete series
\newcommand{\ordser}[2]{%
    \foreach \index in {0, ..., #1} {%
        \ordep{\index}{#2}
    }
    + \ldots
}


% Some old commands
\newcommand{\expp}[1]{\text{e}^{#1}}
\newcommand{\ms}{\frac{\text{m}}{\text{s}}}
\newcommand{\mms}{\frac{\text{m}^2}{\text{s}}}
\newcommand{\muo}{\mu_\circ}
\newcommand{\re}[1]{#1_{\text{Re}}}
\newcommand{\im}[1]{#1_{\text{Im}}}
\providecommand\Div{\mathop{\rm div}\nolimits}                  %%%%%%%%%
\providecommand\Rot{\mathop{\rm rot}\nolimits}                  %%%%%%%%%
\providecommand\Grad{\mathop{\rm grad}\nolimits}                %%%%%%%%%
\providecommand\Grad{\mathop{\rm grad}\nolimits}                %%%%%%%%%


\DeclareMathOperator{\tg}{tg}
\DeclareMathOperator{\arctg}{arctg}

% Tensoren fett
\DeclareRobustCommand{\tensor}[1]{{\mbox{\mathversion{bold}\ensuremath{#1}}}}

\def\inputGnumericTable{}

\renewcommand{\figurename}{Fig.}

\begin{document}

\sloppy

\section{General information}

This software is a script compilation for FreeFem++ to solve interface cell problems and arrive
with coefficients in boundary condition (BC) between porous or textured medium and free fluid.
This module gives the possibility to compute the coefficients for BC in two-dimensions (2D) for
porous case. If you find this software useful, please cite the corresponding
publication <DOI:10.1017/jfm.2016.838> (open access) This software is also
suitable to determine coefficients in TR model (improved accuracy over the original work), which
is available as a pre-print <arXiv:1812.09401>.

The contents of this module are listed below.
%[backgroundcolor=\color{black},stringstyle=\color{green},keywordstyle=\color{green}]
\begin{lstlisting}[language=tex]
src/2D_MSE_bc_solver.edp                   - script to solve for BC coefficients tensor
src/2D_MSE_solver.edp                      - script to solve a Cavity problem with
                                              developed boundary conditions
doc/2D_MSE_bc_velocity.pdf                 - this document
\end{lstlisting}

Two additional .idp files are provided but not described in detail. In this document we explain the key elements of the software, parameters of the simulation, and implemented variational formulation.

\section{Solver for interface tensors K and L}

\subsection{Structure of the simulation script}

The simulation script take following input:
\begin{enumerate}
    \item geometry of the interface cell (through predefined pre-sets or custom definition);
	\item interface location ($y_i$);
    \item extent of the cell (set by number of structures $Nstr$);
    \item solid volume fraction ($thetas$) and other structure parameters;
\end{enumerate}

Currently the geometry is generated within the FreeFem++. The mesh resolution is
controlled by parameters $n$ -- segments per unit length --, $nIntf$ -- 
refinement at interface -- and $nS$ -- refinement at solid structures.

Currently, there are two pre-sets defined for the textured surfaces and three pre-sets
defined for porous surfaces. To switch between textured and porous surfaces, one has to
modify the macro
\begin{lstlisting}[numbers=left,firstnumber=46]
macro Type()textured// EOM                                                                    
\end{lstlisting}
which can take values ``textured'' or ``porous''.
Two textured pre-sets available are ``tricav'' and ``rctcav'', which have been
tested in  <arXiv:1812.09401>. The porous pre-sets are
``cylinder'' (isotropic cylinder
configuration), ``laycyl'' (layered cylinder configuration) and ``ellipse'' (tilted ellipse configuration).
The first one corresponds to the original work <DOI:10.1017/jfm.2016.838>, while the second two were
in addition considered in <arXiv:1812.09401>. The switching between the presets
are implemented using the macro ``geomPreset'' as
\begin{lstlisting}[numbers=left,firstnumber=47]
macro geomPreset()rctcav// EOM                                                                
real wdth = 0.5;
real dpth = 0.5;
\end{lstlisting}
followed by the needed geometrical parameters.
Note that there can be no white space between name defined by the macro and end-of-macro (comment symbols
``//''). The geometry generation relies on use of macros and IFMACRO statement, introduced in relatively
recent FreeFEM++ version, 3.51 and above. More examples of geometry definition can be found in
example file ``examples\_geomDef.edp''. The convention of all geometries is that the upper most solid
point has coordinate $y = 0$. For structures, where the upper surface has planar component, it is possible
to set interface coordinate to $y_i = 0$. For curved surfaces it is not possible do to practical reasons.
The zero interface location is indicated with the macro
\begin{lstlisting}[numbers=left,firstnumber=60]
macro itfFlag()zero// EOM // Interface flag, "above" or "zero"                                
\end{lstlisting}
Assertion is carried out for proper definition of the interface location (lines 63--64).
Additionally, for the porous set-ups, the classical interior permeability problem 
is defined and solved in include file ``include\_interK.idp'' and
used in the main solver, see lines 76--82.
The solution is later used for
getting appropriate boundary conditions for the interface cell 
see <DOI:10.1017/jfm.2016.838>), the results are stored in
separate finite element functions.

The result of the simulation script is tensor fields at the full interface cell. The plane average result is
outputted in text files. The fields are post-processed according to <DOI:10.1017/jfm.2016.838>
and <arXiv:1812.09401>; resulting boundary condition coefficients are outputted to the console and
saved to text files.

\newpage
\subsection{Variational formulation}

The developed software solve coupled systems of Stoke's equations. By fixing indices $k,l$, we set $u_i^{K^{\pm}} = K^{\pm}_{ik}$, $u_i^{L^{\pm}} = L^{\pm}_{ikl}$, $p^{K^{\pm}} = A^{\pm}_{k}$ and $p^{L^{\pm}} = B^{\pm}_{kl}$. Then the governing equations take form (see paper<DOI:10.1017/jfm.2016.838> for all details)
\begin{align}
- p^{K^{+}}_{,i} + u^{K^{+}}_{i,jj} & = 0, &  - p^{L^{+}}_{,i} + u^{L^{+}}_{i,jj} & = 0, \\
u^{K^{+}}_{i,i} & = 0, & u^{L^{+}}_{i,i} & = 0 , \\
\left. u^{K^{+}}_{i} \right|_{\Gamma_C} & = 0, & \left. u^{L^{+}}_{i} \right|_{\Gamma_C} & = 0, \\
\left. u^{K^{-}}_{i} \right|_{\Gamma_I} & = \left. u^{K^{+}}_{i} \right|_{\Gamma_I}, &  \left. u^{L^{-}}_{i} \right|_{\Gamma_I} & = \left. u^{L^{+}}_{i} \right|_{\Gamma_I},  \\
[ - p^{K^{+}} \delta_{ij} + 2 & \left. \str{}{ij}{u^{K^{+}}} ] \right|_{\Gamma_I} n_j = &  [ - p^{L^{+}} \delta_{ij} + 2 & \left. \str{}{ij}{u^{L^{+}}} ] \right|_{\Gamma_I} n_j = \\
= [ - p^{K^{-}} \delta_{ij} + 2 & \left. \str{}{ij}{u^{K^{-}}} ] \right|_{\Gamma_I} n_j,  &  = [ - p^{L^{-}} \delta_{ij} + 2 & \left. \str{}{ij}{u^{L^{-}}} ] \right|_{\Gamma_I} n_j - \delta_{ik} n_l,  \\
- p^{K^{-}}_{,i} + u^{K^{-}}_{i,jj} & = - \delta_{ik},  & - p^{L^{-}}_{,i} + u^{L^{-}}_{i,jj} & = 0,  \\
u^{K^{-}}_{i,i} & = 0, & u^{L^{-}}_{i,i} & = 0, \\
\left. u^{K^{-}}_{i} \right|_{\Gamma_C} & = 0, & \left. u^{L^{-}}_{i} \right|_{\Gamma_C} & = 0.
\end{align}

Structure of both set of equation systems is exactly the same, the only difference lies in volume forcing and stress boundary condition. Without loosing generality, we proceed by showing how to solve problems for $K$. We also do not implement the forcing directions for the $L$ problem which are zero by construction. In order to simplify notation, we skip the $K$ in equations. To solve these problems, we define test functions $\hat{u}_i^{\pm}$ and $\hat{p}^{\pm}$ for velocity and pressure. To arrive with weak formulation, we multiply equations with test function and employ integration by parts for fluid stresses, designated by $\Sigma^{\pm}_{ij} = - p^{\pm} \delta_{ij} + 2 \str{}{ij}{u^{\pm}} $, to obtain
\begin{align}
- \int_{\Omega} \Sigma^{+}_{ij} \hat{u}^{+}_{i,j} \,\mathit{d\Omega} + \int_{\pd \Omega} \Sigma^{+}_{ij} n_j \hat{u}^{+}_{i} \,\mathit{dS} & = 0, \\
\int_{\Omega} u^{{+}}_{i,i} \hat{p}^{+} \,\mathit{d\Omega} & = 0, \\
- \int_{\Omega} \Sigma^{-}_{ij} \hat{u}^{-}_{i,j} \,\mathit{d\Omega} + \int_{\pd \Omega} \Sigma^{-}_{ij} n_j \hat{u}^{-}_{i} \,\mathit{dS} & = -\int_{\Omega} \delta_{ik} \hat{u}^{-}_{i} \,\mathit{d\Omega},  \\
\int_{\Omega} u^{{-}}_{i,i} \hat{p}^{-} \,\mathit{d\Omega} & = 0,
\end{align}
for governing equations. This weak formulation is implemented using problem statements (lines 113--124 and 161--166). The surface
forcing for the slip problem is enforced using surface integral on the interior boundary.

\section{Solver for led driven cavity flow over porous medium}

The simulation script take following input:
\begin{enumerate}
    \item geometry of the flow problem, both free fluid and porous medium, number of grid points $n$ per unit length;
	\item scale separation parameter $epsP$ and volume fraction $thetas$;
    \item interface location ($yif$);
    \item results from interface cell (tensors $K$ and $L$);
    \item Upper wall velocity $Udrv$.
\end{enumerate}

The domain consists of free fluid domain $\Omega_f$ and porous medium domain $\Omega_p$, encompassed by boundaries at interface $\Gamma_{if}$, upper wall $\Gamma_{f2}$ and side walls $\Gamma_{f13}$ of free fluid, and lower wall $\Gamma_{p4}$ and side walls $\Gamma_{p13}$ of porous domain. Within these two domains, we solve equation system
\begin{align}
- p_{,i} + \ord u_{i,jj} & = 0 & \mbox{ in } \Omega_f, & & \left. u_i \right|_{\Gamma_{f13}} & = 0, \ \ \ \ \left. u_1 \right|_{\Gamma_{f2}} = Udrv, \ \ \ \ \left. u_2 \right|_{\Gamma_{f2}} = 0, \\
u_{i,i} & = 0 & \mbox{ in } \Omega_f, & & \left. u_1 \right|_{\Gamma_{if}} & = - K_{1j} \frac{\tilde{p}_{,j}}{\ord} + L_{1ij} \left( u_{i,j} + u_{j,i} \right), \\
& & & & \left. u_2 \right|_{\Gamma_{if}} & = - K_{2j} \frac{\tilde{p}_{,j}}{\ord} + L_{2ij} \left( u_{i,j} + u_{j,i} \right), \\
\tilde{p}_{,jj} & = 0 & \mbox{ in } \Omega_p, & & \left. \tilde{p} \right|_{\Gamma_{if}} & = p, \ \ \ \ \left. \tilde{p}_{,1} \right|_{\Gamma_{p13}} = 0, \ \ \ \ \left. \tilde{p}_{,2} \right|_{\Gamma_{p4}} = 0.
\end{align}
The Darcy flow in the interior is expressed using the Darcy law with constant isotropic permeability $K_{drc}$
\begin{equation}
\tilde{u}_i = - K_{drc} \frac{\tilde{p}_{,i}}{\ord} .
\end{equation}
The weak form for Stokes equations are obtained as explained in the
previous section, while for the Darcy region the weak for is obtained
as for a standard Poisson equation. The boundary conditions at the
coupled interface are ensured by help of Lagrange multipliers, i.e.,
additional test function $\hat{\lambda}$ is introduced and
boundary conditions are multiplied with the function and integrated
over the surface. The feedback to governing equations has been empirically
determined. The forcing term from boundary conditions of velocity acts
only on velocities (and not on derivatives
of velocity or pore pressure), while forcing from pressure boundary
condition for pore
pressure equation acts only in the pore pressure equation.
We have also empirically determined the resolution for Langrange multiplier
space and the finite element order, see the main simulation file.

\end{document}
